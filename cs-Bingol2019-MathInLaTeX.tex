% !TEX spellcheck = en_US

\documentclass[a4paper,10pt,landscape]{article}

\usepackage[utf8]{inputenc} % To use Unicode characters
\usepackage[iso]{datetime}
	\newcommand{\hbTimeStamp}{{\color{red}v\today/\currenttime}} % version



\usepackage{etex}  % prevent counter errors




% ======================================= V
%:  AMS packages 
\usepackage{amssymb}
\usepackage{amsmath}
\usepackage{amsfonts}
\usepackage{amsthm}
\newtheorem{thm}{Theorem}[section]
%
\newtheorem{cor}[thm]{Corollary}
\newtheorem{lem}[thm]{Lemma}
\newtheorem{prop}[thm]{Proposition}
\newtheorem{ax}{Axiom}
%
\theoremstyle{definition}
\newtheorem{defn}{Definition}[section]
%
\theoremstyle{remark}
\newtheorem{rem}{Remark}[section]
\newtheorem*{notation}{Notation}
\newtheorem{exmp}{Example}[section] % @HB
% ======================================= A





% ======================================= V
%: references
\newcommand{\reffig}[1]{Fig.~\ref{#1}}
\newcommand{\reftbl}[1]{Table~\ref{#1}}
\newcommand{\refsec}[1]{Sec.~\ref{#1}}
\newcommand{\refthm}[1]{Theorem~\ref{#1}}
\newcommand{\reflem}[1]{Lemma~\ref{#1}}
\newcommand{\refdef}[1]{Definition~\ref{#1}}
\newcommand{\refeq}[1]{Eq.~\ref{#1}}
% ======================================= A

%
\usepackage{cancel}
\usepackage{multicol}
\usepackage{enumerate}

\usepackage{color}
	\definecolor{darkred}{rgb}{0.8,0.1,0.1}
	\definecolor{darkgreen}{rgb}{0,0.5,0}
	\definecolor{darkblue}{rgb}{0,0,0.5}

\usepackage[all]{xy}
	\CompileMatrices % \xymatrix


% ===========
\newcommand{\hDif}{\mathrm{d}}			% differential dx
\newcommand{\hDefined}[1]{\textcolor{darkred}{\textit{#1}}} % definition
\newcommand{\hDefinitionIFF}{\ensuremath{ \ \overset{\Delta}{\longleftrightarrow}} \ } % <-->
\newcommand{\hDefinitionEqual}{\ensuremath{ \ \triangleq} \ } % =
\newcommand{\hDefinitionEquiv}{\ensuremath{ \ \overset{\Delta}{\equiv}} \ } % =
\newcommand{\hAbs}[1]{\ensuremath{\left \lvert \, #1 \, \right \rvert} } % |x|
\newcommand{\hCeil}[1]	{\ensuremath{\lceil  \, #1 \,  \rceil }} % ceil(x)
\newcommand{\hFloor}[1]	{\ensuremath{\lfloor  \, #1 \,  \rfloor }} % floor(x)

% ==== HB Header Common Sets v20181201 ====V
\newcommand{\hSoN}  {\ensuremath{\mathbb{N}}}      % set of Natural Numbers
\newcommand{\hSoNp} {\ensuremath{\mathbb{N}^{+}}} % set of Natural +
\newcommand{\hSoZ}  {\ensuremath{\mathbb{Z}}}      % set of Integers
\newcommand{\hSoZp} {\ensuremath{\mathbb{Z}^{+}}} % set of Integers +
\newcommand{\hSoZn} {\ensuremath{\mathbb{Z}^{-}}} % set of Integers -
\newcommand{\hSoZnn}{\ensuremath{\mathbb{Z}_{\ge 0}}} % set of Integers non negative
\newcommand{\hSoZnz}{\ensuremath{\mathbb{Z}_{\neq 0}}} % set of Real +
\newcommand{\hSoZnp}{\ensuremath{\mathbb{Z}_{\le 0}}} % set of Integers non positive
\newcommand{\hSoQ}  {\ensuremath{\mathbb{Q}}}      % set of Rationals 
\newcommand{\hSoQp} {\ensuremath{\mathbb{Q}^{+}}} % set of Rationals +
\newcommand{\hSoQn} {\ensuremath{\mathbb{Q}^{-}}} % set of Rationals -
\newcommand{\hSoQnn}{\ensuremath{\mathbb{Q}_{\ge 0}}} % set of Rationals non negative
\newcommand{\hSoQnz}{\ensuremath{\mathbb{Q}_{\neq 0}}} % set of Real +
\newcommand{\hSoQnp}{\ensuremath{\mathbb{Q}_{\le 0}}} % set of Rationals non positive
\newcommand{\hSoR}  {\ensuremath{\mathbb{R}}}      % set of Real Numbers
\newcommand{\hSoRp} {\ensuremath{\mathbb{R}^{+}}} % set of Real +
\newcommand{\hSoRn} {\ensuremath{\mathbb{R}^{-}}} % set of Real -
\newcommand{\hSoRnn}{\ensuremath{\mathbb{R}_{\ge 0}}} % set of Reals non negative
%\newcommand{\hSoRnp}{\ensuremath{\mathbb{R}_{\ge 0}}} % set of Reals non positive
\newcommand{\hSoRnz}{\ensuremath{\mathbb{R}_{\neq 0}}} % set of Real +
\newcommand{\hSoRnp}{\ensuremath{\mathbb{R}_{\le 0}}} % set of Reals non positive
\newcommand{\hSoC}  {\ensuremath{\mathbb{C}}}      % set of Complex Numbers
\newcommand{\hSoCnz}{\ensuremath{\mathbb{C}_{\neq 0}}} % set of nonzero Complex Numbers
%
\newcommand{\hSoPrimes}{\ensuremath{\mathbb{P}}} % set of Prime Numbers
\newcommand{\hSoTruth}{\ensuremath{\left \{ T, F \right \}}} % {T, F}
%\newcommand{\hSoBinary}{\ensuremath{\left \{ 0, 1 \right \}}} % {0, 1}
\newcommand{\hSoBits}{\ensuremath{\mathbb{B}}} % set of Bits
%
\newcommand{\hSoFunctions}[2]{\ensuremath{#2^{#1}}} % set of functions from 1to2
\newcommand{\hSoSubsets}[1]{\ensuremath{2^{#1}}} % set of subsets of 1
\newcommand{\hSoPowerSet}[1]{\ensuremath{\mathcal{P}(#1)}} % power set of 1
% ==== HB Header Common Sets v20181201 ====A


%\usepackage[a4paper,landscape]{geometry}

%\usepackage{mathtools}


%
%\usepackage{multicol}
%\usepackage{calc}
%\usepackage[landscape]{geometry} % landscape


% To Do:
% \listoffigures \listoftables
% \setcounter{secnumdepth}{0}

% This stuff fails to set the page size properly for some reason

%\usepackage{anysize}
%\marginsize{0.5in}{0.5in}{0.5in}{0.5in}

%\special{papersize=11in,8.5in}
%\setlength{\textwidth}{10in}
%\setlength{\textheight}{6.5in}
%\marginsize{0.5in}{0.5in}{0.5in}{0.5in}


% All this page size stuff is a hack, but it seems to work
% Turn off header and footer
\pagestyle{empty}

\setlength{\oddsidemargin}{0.25in}
\setlength{\evensidemargin}{0.5in}
\setlength{\textwidth}{10in}


\setlength{\topmargin}{-0.75in}
\setlength{\textheight}{7.25in}
\setlength{\headheight}{0in}
\setlength{\headsep}{0in}



 

% Redefine section commands to use less space
\makeatletter
\renewcommand{\section}{\@startsection{section}{1}{0mm}%
                                {-1ex plus -.5ex minus -.2ex}%
                                {0.5ex plus .2ex}%x
                                {\normalfont\large\bfseries}}
\renewcommand{\subsection}{\@startsection{subsection}{2}{0mm}%
                                {-1explus -.5ex minus -.2ex}%
                                {0.5ex plus .2ex}%
                                {\normalfont\normalsize\bfseries}}
\renewcommand\subsubsection{\@startsection{subsubsection}{3}{0mm}%
                                {-1ex plus -.5ex minus -.2ex}%
                                {1ex plus .2ex}%
                                {\normalfont\small\bfseries}}
\makeatother

% Define BibTeX command
\def\BibTeX{{\rm B\kern-.05em{\sc i\kern-.025em b}\kern-.08em
    T\kern-.1667em\lower.7ex\hbox{E}\kern-.125emX}}

% Don't print section numbers
\setcounter{secnumdepth}{1}
%\setcounter{secnumdepth}{0}


\setlength{\parindent}{0pt}
\setlength{\parskip}{0pt plus 0.5ex}


% -----------------------------------------------------------------------

\begin{document}

\raggedright
\footnotesize
\begin{multicols}{3}


% multicol parameters
% These lengths are set only within the two main columns
%\setlength{\columnseprule}{0.25pt}
\setlength{\premulticols}{1pt}
\setlength{\postmulticols}{1pt}
\setlength{\multicolsep}{1pt}
\setlength{\columnsep}{2pt}

\begin{center}
     \Large{\textbf{
     	Cheat Sheet\\
	Math in \LaTeX
     }}\\
     \small{\hbTimeStamp}
\end{center}




% ======================================================================
\section{Basics}
\label{sec:basics}
\begin{tabular}{@{}ll@{}}
%	$\mathbf{a}$	&\verb!\mathbf{a}!\\
	$a, \pmb{a}$	&\verb!a, \pmb{a}!\\
%	$\Phi, \boldsymbol{\Phi}$	&\verb!\Phi, \boldsymbol{\Phi}!\\
	$\Phi, \pmb{\Phi}$	&\verb!\Phi, \pmb{\Phi}!\\
	$a^{ij}$	&\verb!a^{ij}!\\
	$a_{ij}$	&\verb!a_{ij}!\\
	$a_{i}^{j}$	&\verb!a_{i}^{j}!\\
	${}_{k}^{l}a_{i}^{j}$	&\verb!{}_{k}^{l}a_{i}^{j}!\\
%	$\head x$	&\verb!\dot x!\\
	$\bar x$	&\verb!\bar x!\\
	$\vec x$	&\verb!\vec x!\\
	$\dot x$	&\verb!\dot x!\\
	$\tilde{a}$	&\verb!\tilde{a}!\\
	%
	$\lvert a \rvert$ (\refsec{sec:pairs})	&\verb!\lvert a \rvert!\\
	$\lceil a \rceil$ (\refsec{sec:pairs})	&\verb!\lceil a \rceil!\\
	$\lfloor a \rfloor$ (\refsec{sec:pairs})	&\verb!\lfloor a \rfloor!\\
	%
	$\overline{ab}$	&\verb!\overline{ab}!\\
	$\underline{ab}$	&\verb!\underline{ab}!\\
	$\overrightarrow{AB}$	&\verb!\overrightarrow{AB}!\\
	$\overleftarrow{AB}$	&\verb!\overleftarrow{AB}!\\
	$\widetilde{ab}$	&\verb!\widetilde{ab}!\\
	$\overbrace{a_{1} \cdots a_{k}}^{k}$	&\verb!\overbrace{a_{1} \cdots a_{k}}^{k}!\\
	$\underbrace{a_{1} \cdots a_{k}}_{k}$	&\verb!\underbrace{a_{1} \cdots a_{k}}_{k}!\\
	$\operatorname*{function}(x)$	&\verb!\operatorname*{function}(x)!\\
	$M^{\top}$	&\verb!M^{\top}!\\
	$\ell_{\bot}$	&\verb!\ell_{\bot}!\\
	$\ell_{\parallel}$	&\verb!\ell_{\parallel}!\\
%	$\underbracket{x}_{\text{real}}$	&\verb!aaa!\\
%	$\underbrace{x}_\text{real}$	&\verb!aaa!\\
%	$aaa$	&\verb!aaa!\\
\end{tabular}





% ======================================================================
\section{Cancellation}
\label{sec:cancellation}
\begin{tabular}{@{}ll@{}}
	$a \neq b$	&\verb!a \neq b!\\
	$a \not\subseteq b$	&\verb!a \not\subseteq b!\\
	$\not aa$	&\verb!\not aa!\\
	$\not{aa}$	&\verb!\not{aa}!\\
	$\cancel{a b c \beta}$	&\verb!\cancel{a b c \beta}!\\	% req. cancel
	$\bcancel{a b c \beta}$	&\verb!\bcancel{a b c \beta}!\\	% req. cancel
	$\xcancel{a b c \beta}$	&\verb!\xcancel{a b c \beta}!\\	% req. cancel
	$\cancelto{\infty}{a b c \beta}$	&\verb!\cancelto{\infty}{a b c \beta}!\\	% req. cancel
%	$aaa$	&\verb!aaa!\\
\end{tabular}




% ======================================================================
\section{Meta}
\label{sec:meta}
\begin{tabular}{@{}ll@{}}
	$x \Longrightarrow y$	&\verb!x \Longrightarrow y!\\
	$x \Longleftarrow y$	&\verb!x \Longleftarrow y!\\
	$x \iff y$	&\verb!x \iff y!\\
%	$aaa$	&\verb!aaa!\\
\end{tabular}




% ======================================================================
\section{More}
\begin{tabular}{@{}ll@{}}
	%
	$\sideset{_{a}^{b}}{_{c}^{d}} \sum$	
	&\verb!\sideset{_{a}^{b}}{_{c}^{d}} \sum!\\
	%
	$\overset{abc}{xyz}$	
	&\verb!\overset{abc}{xyz}!\\
	%
	$\underset{i}{\arg \, \min} \, \{ a_{i} \}$	
	&\verb!\underset{i}{\arg \, \min} \, \{ a_{i} \}!\\
	%
	$\underset{a}{\arg \, \max} \, f(a)$	
	&\verb!\underset{a}{\arg \, \max} \, f(a)!\\
	%
	$\frac{a}{b}$	
	&\verb!\frac{a}{b}!\\
	%
	$\tfrac{a}{b}$	
	&\verb!\tfrac{a}{b}!\\
	%
	$\sum_{i = 0}^{5} i$	
	&\verb!\sum_{i = 0}^{5} i!\\
	%
	$\prod_{i = 0}^{5} i$	
	&\verb!\prod_{i = 0}^{5} i!\\
	%
	$\lim_{a \to \infty} x$	
	&\verb!\lim_{a \to \infty} x!\\
	%
		  $\frac{\hDif f(x)}{\hDif x}$	
	&\verb!\frac{\hDif f(x)}{\hDif x}!\\
	%
	$a + \mathrm{i}\, b$
	&\verb~a + \mathrm{i}\, b~\\
	%
		  $\int_{a}^{b} \! f(x) \, \hDif x$	
	&\verb~\int_{a}^{b} \! f(x) \, \hDif x~\\
	%
	$\left[\frac{\infty}{\infty}\right]$ 
	&\verb~\left[\frac{\infty}{\infty}\right]~\\
%\makeatletter
%\renewcommand\d[1]{\mspace{6mu}\mathrm{d}#1\@ifnextchar\d{\mspace{-3mu}}{}}
%\makeatother
%	$aaa$	&\verb!aaa!\\
\end{tabular}




% =================
\section{Format Patterns}
% ----------------------------
%\begin{tabular}{@{}ll@{}}
%	a
%	\begin{tabular}{|l|}
%		b \\
%		c
%	\end{tabular}
%	d.\\
%&\verb!a
%	\begin{tabular}{|l|}
%		b \\
%		c
%	\end{tabular}
%	d.\\
%!\\
%\end{tabular}\\

\begin{multicols}{2}
	a
	\begin{tabular}{|l|}
		b \\
		c
	\end{tabular}
	d.\\
\columnbreak
	\begin{verbatim}
		a
		\begin{tabular}{|l|}
			  b \\
			  c
		\end{tabular}
		d.
	\end{verbatim}
\end{multicols}
% ----------------------------
\begin{multicols}{2}
	 ${\genfrac(]{0pt}{2}{a+b}{c+d+e}}$
\columnbreak
	\begin{verbatim}
		 ${\genfrac(]{0pt}{2}
		 {a+b}
		 {c+d+e}}$
	\end{verbatim}
\end{multicols}

% ----------------------------
\begin{multicols}{2}
	$a = 
	\begin{cases}
		1, & n \text{ is odd}, \\
		0, & \text{otherwise}.
	\end{cases}$
\columnbreak
	\begin{verbatim}
	$a = 
	\begin{cases}
		  1, & n \text{ is odd}, \\
		  0, & \text{otherwise}.
	\end{cases}$
	\end{verbatim}
\end{multicols}
% ----------------------------
\begin{multicols}{2}
	\[\sum_{\substack{k \in \hSoZ\\ 7 < k \\ k \leq 4}} a(k).\]
\columnbreak
	\begin{verbatim}
	  $\sum_{
	      \substack{
	        k \in \hSoZ\\ 
	        7 < k \\ 
	        k \leq 4
	      }
	    } 
	    a(k).$
	\end{verbatim}
\end{multicols}
% ----------------------------
\begin{multicols}{2}
	\[n! = \underbrace{1 \cdot 2 \cdot \dotso \cdot n}_{n}.\]
\columnbreak
	\begin{verbatim}
		n! = \underbrace{
		         1 
		         \cdot 2 
		         \cdot 
		         \dotso 
		         \cdot n
		      }_{n}.
	\end{verbatim}
\end{multicols}
% ----------------------------
\begin{multicols}{2}
		\begin{align*}
			aaa &= b+b+b &// ccc\\
			d &= e+e &// fff
		\end{align*}
\columnbreak
	\begin{verbatim}
		\begin{align*}
			aaa &= b+b+b &// ccc\\
			d &= e+e &// fff
		\end{align*}
	\end{verbatim}
\end{multicols}



% =================
\section{Equations}
% ----------------------------
\begin{multicols}{2}
	\begin{align*}
		1 + (2 + 3)
		& = 1 + 5\\
		& = 6\\
		& = 12/2.
	\end{align*}
\columnbreak
	\begin{verbatim}
	\begin{align*}
		1 + (2 + 3)
		& = 1 + 5\\
		& = 6\\
		& = 12/2.
	\end{align*}
	\end{verbatim}
\end{multicols}
% ----------------------------
\begin{multicols}{2}
	\[
		\left.
			\frac{x^{2}}{3}
		\right|_{0}^{1}
	\]
\columnbreak
	\begin{verbatim}
		\left.
			 \frac{x^{2}}{3}
		\right|_{0}^{1}
	\end{verbatim}
\end{multicols}
%% ----------------------------
%\begin{multicols}{2}
%	aaa
%\columnbreak
%	\begin{verbatim}
%		aaa
%	\end{verbatim}
%\end{multicols}







% =======================================
\section{Dots}
\begin{tabular}{@{}ll@{}}
	$1 \cdot 2 \cdot 3$ (multiplication) & \verb!$1 \cdot 2 \cdot 3$!\\
	$1, 2, \dotsc, 9$ (comma) & \verb!$1, 2, \dotsc, 9$!\\
	$1 + 2 + \dotso + 9$ (operator) & \verb!$1 + 2 + \dotso + 9$!\\
\end{tabular}
% ---
\begin{multicols}{2}
	\[
%	 A =
	 \begin{pmatrix}
	  a_{1,1} & \cdots & a_{1,n}\\
	  \vdots  & \ddots & \vdots \\
	  a_{m,1} & \cdots & a_{m,n}
	 \end{pmatrix}
	\]
\columnbreak
	\begin{verbatim}
		$\begin{pmatrix}
		  a_{1,1} &\cdots &a_{1,n} \\
		  \vdots  &\ddots &\vdots  \\
		  a_{m,1} &\cdots &a_{m,n}
		 \end{pmatrix}$
	\end{verbatim}
\end{multicols}
% ---
\begin{multicols}{2}
\[
M = \bordermatrix{~ & x & y \cr
                  A & 1 & 2 \cr
                  B & 3 & 4 \cr}
\]
\columnbreak
	\begin{verbatim}
	$M = \bordermatrix{~ & x & y \cr
                  A & 1 & 2 \cr
                  B & 3 & 4 \cr}
	$
	\end{verbatim}
\end{multicols}

%% ---
%\begin{multicols}{2}
%	aaa
%\columnbreak
%	\begin{verbatim}
%	aaa
%	\end{verbatim}
%\end{multicols}



% =================
\section{Logic}
\begin{tabular}{@{}ll@{}}
	$\overline{p}$	&\verb!\overline{p}!\\
	$\neg p$	&\verb!\neg p!\\
	$p \land q$	&\verb!p \land q!\\
	$p \lor q$	&\verb!p \lor q!\\
	$p \oplus q$	&\verb!p \oplus q!\\
	$p \rightarrow q$	&\verb!p \rightarrow q!\\
	$p \leftrightarrow q$	&\verb!p \leftrightarrow q!\\
	$p \equiv q$	&\verb!p \equiv q!\\
%--
%	$p \hAnd q$	&\verb!p \hAnd q!\\
%	$p \hOr q$	&\verb!p \hOr q!\\
	$p \longrightarrow q$	&\verb!p \longrightarrow q!\\
	$p \longleftrightarrow q$	&\verb!p \longleftrightarrow q!\\
	$p \Longrightarrow  q$	&\verb!p \Longrightarrow  q!\\
	$p \Longleftrightarrow  q$	&\verb!p \Longleftrightarrow  q!\\
\end{tabular}
\begin{tabular}{@{}ll@{}}
	$\forall x \in A \ P(x)$	&\verb!\forall x \in A \ P(x)!\\
	$\not \forall x \in A \ P(x)$	&\verb!\not \forall x \in A \ P(x)!\\
	$\exists x \in A \ P(x)$	&\verb!\exists x \in A \ P(x)!\\
	$\not \exists x \in A \ P(x)$	&\verb!\not \exists x \in A \ P(x)!\\
%	$aaa$	&\verb!aaa!\\
\end{tabular}



% =================
\section{Set Theory}
\begin{tabular}{@{}ll@{}}
%	$\hPairing{b}{aa}$	&\verb!\hPairing{b}{aa}!\\
%	$\ensuremath{\left 1 \, 2 \,  2}$	&aaa\\
	$\emptyset$	&\verb!\emptyset!\\
	$x \in A$	&\verb!x \in A!\\
	$x \notin A$	&\verb!x \notin A!\\
	$\{ x, y \}$	& \verb!\{ x, y \}!\\
	$\{ x \mid P(x) \}$	&\verb!\{ x \mid P(x) \}!\\
	$A \subset B$	&\verb!A \subset B!\\
	$A \not\subset B$	&\verb!A \not\subset B!\\
	$A \subseteq B$	&\verb!A \subseteq B!\\
	$A \not\subseteq B$	&\verb!A \not\subseteq B!\\
	$2^{A}$	&\verb!2^{A}!\\
	$\lvert A \rvert$ (\refsec{sec:pairs})	&\verb!\lvert A \rvert!\\
	$A \cup B$	&\verb!A \cup B!\\
	$A \cap B$	&\verb!A \cap B!\\
	$A \smallsetminus B$	&\verb!A \smallsetminus B!\\
	$A \times B$	&\verb!A \times B!\\
	$(a, b)$	&\verb!(a, b)!\\
%	
	$\overline{A}$	&\verb!\overline{A}!\\
	$f^{-1}$	&\verb!f^{-1}!\\
	$f \circ g$	&\verb!f \circ g!\\
	$a \, \beta \, b$	&\verb!a \, \beta \, b!\\
	$a \, \cancel{\beta} \, b$	&\verb!a \, \cancel{\beta} \, b!\\
%	$\hFunction{f}{A}{B}$	&\verb!\hFunction{f}{A}{B}!\\
	$f \colon A \to B$	&\verb!f \colon A \to B!\\
	$a \mapsto f(a)$	&\verb!a \mapsto f(a)!\\
	$f \upharpoonright C$	 (restriction of $f$ to C)	&\verb!f \upharpoonright C!\\
	$i_{A}$	&\verb!i_{A}!\\
%	$aaa$	&\verb!aaa!\\
	$B^{A}$	&\verb!B^{A}!\\
	$\mathcal{P}(A)$	&\verb!\mathcal{P}(A)!\\
	$\mathbb{A}$	&\verb!\mathbb{A}!\\
%	$aaa$	&\verb!aaa!\\
%	$aaa$	&\verb!aaa!\\
%	$aaa$	&\verb!aaa!\\
\end{tabular}



% =================
\section{Algebraic Structures}
\begin{tabular}{@{}ll@{}}
	$[A, \oplus]$	&\verb![A, \oplus]!\\
	$[A, \oplus, \otimes]$	&\verb![A, \oplus, \otimes]!\\
%	$aaa$	&\verb!aaa!\\
\end{tabular}



% =================
\section{Number Theory}
\begin{tabular}{@{}ll@{}}
	$\hCeil{x}$ &\verb!\hCeil{x}!\\
	$\hFloor{x}$ &\verb!\hPairingFloor{x}!\\
	$\hAbs{x}$ &\verb!\hAbs{x}!\\
	$x \mid y$ &\verb!x \mid y!\\
	$x \nmid y$ &\verb!x \nmid y!\\
	$x \bot y$ &\verb!x \bot y!\\
	$x \mathrm{div} y$ &\verb!x  \mathrm{div} y!\\
	$\log_{2} x$ &\verb!\log_{2} x!\\
%	$H_{2} x$ &\verb!H_{2} x!\\
%	$\hBinaryEntropy$ &\verb!\hBinaryEntropy!\\
	$\sum_{k=1}^{n}$ &\verb!\sum_{k=1}^{n}!\\
	$\prod_{k=1}^{n}$ &\verb!\prod_{k=1}^{n}!\\
	$\frac{x}{y}$ &\verb!\frac{x}{y}!\\
	$\sqrt[n]{x}$ &\verb!\sqrt[n]{x}!\\
	$a \bmod b$ &\verb!a \bmod b!\\
	$0 \equiv 3 \pmod{3}$ &\verb!0 \equiv 3 \pmod{3}!\\
	$0 \equiv 3 \mod{3}$  &\verb!0 \equiv 3 \mod{3}!\\
	$0 \equiv 3 \pod{3}$ &\verb!0 \equiv 3 \pod{3}!\\
	$a \perp a$	&\verb!a \perp a!\\
	$a \bot b$ &\verb!a \bot b!\\
%	$aaa$ &\verb!aaa!\\
\end{tabular}



% =================
\section{Combinatorics}
\begin{tabular}{@{}ll@{}}
	$n! = n$ factorial	&\verb!n!!\\
	$n^{\overline{r}}$ = $n$ to the $r$ rising	&\verb!n^{\overline{r}}!\\
	$n^{\underline{r}}$ = $n$ to the $r$ falling	&\verb!n^{\underline{r}}!\\
	${n \choose r}$ = $n$ choose $r$	&\verb!{n \choose r}!\\
	${n \brack r}$ = Stirling cycle number	&\verb!{n \brack r}!\\
	${n \brace r}$ = Stirling subset number &\verb!{n \brace r}!\\
	$S(n, k)$ = Stirling number	&\verb!S(n, k)!\\
	%=The number of partitions of an $n$-set into exactly $k$ nonempty subsets. \\
	$B_{n}$ = The $n$th Bell number	&\verb!B_{n}!\\
	%=The number of partitions of an $n$-set. \\
\end{tabular}


%% =================
%\section{psmatrix}
%$ \begin{psmatrix}[mnode=circle,colsep=1cm,rowsep=.5cm]
%	A \\
%	& B\times_Z C & [mnode=oval] DDD \\
%	& E & F
%	\psset{arrows=->,nodesep=3pt}
%	\everypsbox{\scriptstyle}
%	\ncline{1,1}{2,2}_{y}
%	\ncline[doubleline=true,linestyle=dashed]{-}{1,1}{2,3}^{x}
%	\ncline[linestyle=dashed]{-}{2,2}{3,2}<{q}
%	\ncline{2,2}{2,3}_{p}
%	\ncarc[arcangle=40,border=3pt]{3,2}{1,1}
%	_[npos=.3]{d}^[npos=.9]{e}
%	\ncline[linestyle=dotted]{2,3}{3,3}>{f}
%	\ncline{3,2}{3,3}^{g}
%\end{psmatrix} $
%\begin{verbatim}
%\begin{psmatrix}[mnode=circle,colsep=1cm,rowsep=.5cm]
%	A \\
%	& B\times_Z C & [mnode=oval] DDD \\
%	& E & F
%	\psset{arrows=->,nodesep=3pt}
%	\everypsbox{\scriptstyle}
%	\ncline{1,1}{2,2}_{y}
%	\ncline[doubleline=true,linestyle=dashed]{-}{1,1}{2,3}^{x}
%	\ncline[linestyle=dashed]{-}{2,2}{3,2}<{q}
%	\ncline{2,2}{2,3}_{p}
%	\ncarc[arcangle=40,border=3pt]{3,2}{1,1}
%	_[npos=.3]{d}^[npos=.9]{e}
%	\ncline[linestyle=dotted]{2,3}{3,3}>{f}
%	\ncline{3,2}{3,3}^{g}
%\end{psmatrix} 
%\end{verbatim}



% =================
\section{xymatrix}
$
	\xymatrix{
	  11
		\POS[];
		[d]**\dir{-};
		[];[dr]**\dir{-};
	\ar@/^/[r]^{\alpha_{1}} 
	& 12
	\ar@/^/[r]^{\alpha_{7}} 
	\ar@/^/[l]^{\alpha_{3}} 
	& 13
	  \ar@/^/[l]^{\alpha_{9}} 
	\\
	  21
	  \ar@/^/[ur]^{\alpha_{5}}
	  \ar@/^/[u]^{\alpha_{4}}
	  \ar@/^/[urr]^{\alpha_{7}}
	& \bullet
	& 23
	  \ar@/^/[ll]^{\alpha_{8}}
	  \ar[u]^{9}
	  \ar[ul]^{9}
	}
$
\begin{verbatim}
$
	\xymatrix{
	  11
		\POS[];
		[d]**\dir{-};
		[];[dr]**\dir{-};
	\ar@/^/[r]^{\alpha_{1}} 
	& 12
	\ar@/^/[r]^{\alpha_{7}} 
	\ar@/^/[l]^{\alpha_{3}} 
	& 13
	  \ar@/^/[l]^{\alpha_{9}} 
	\\
	  21
	  \ar@/^/[ur]^{\alpha_{5}}
	  \ar@/^/[u]^{\alpha_{4}}
	  \ar@/^/[urr]^{\alpha_{7}}
	& \bullet
	& 23
	  \ar@/^/[ll]^{\alpha_{8}}
	  \ar[u]^{9}
	  \ar[ul]^{9}
	}
$
\end{verbatim}
% ---
%\begin{multicols}{2}
%	\begin{hSolution}
%		aa
%	\end{hSolution}
%
%\columnbreak
%	\begin{verbatim}
%	\begin{hSolution}
%	 aa
%	\end{hSolution}
%	\end{verbatim}
%\end{multicols}
%% ---
%\begin{multicols}{2}
%	\begin{hNotation}
%		aa
%	\end{hNotation}
%
%\columnbreak
%	\begin{verbatim}
%	\begin{hNotation}
%		aa
%	\end{hNotation}
%	\end{verbatim}
%\end{multicols}



% =================
\section{Math Templates}
\label{sec:templates}
\begin{verbatim}
	\newcommand{\reffig}[1]{Fig.~\ref{#1}}
	\newcommand{\reftbl}[1]{Table~\ref{#1}}
	\newcommand{\refsec}[1]{Sec.~\ref{#1}}
	\newcommand{\refthm}[1]{Theorem~\ref{#1}}
	\newcommand{\reflem}[1]{Lemma~\ref{#1}}
	\newcommand{\refdef}[1]{Definition~\ref{#1}}
	\newcommand{\refeq}[1]{Eq.~\ref{#1}}
\end{verbatim}
% ----------------------------
\begin{multicols}{2}
	\begin{ax}
		aa 
	\end{ax}
\columnbreak
	\begin{verbatim}
	\begin{ax}
		aa 
	\end{ax}
	\end{verbatim}
\end{multicols}

% ----------------------------
\begin{multicols}{2}
	\begin{defn}
		aa \hDefined{dd} aa.
		\label{def:aa}
	\end{defn}
\columnbreak
	\begin{verbatim}
		\begin{defn}
			Aaaa
			\label{def:aa}
		\end{defn}
	\end{verbatim}
\end{multicols}

% ----------------------------
\begin{multicols}{2}
	\begin{thm}[Gauss]
		aa	
		\label{thm:gauss}
	\end{thm}
\columnbreak
	\begin{verbatim}
	\begin{thm}[Gauss]
		aa	
		\label{thm:gauss}
	\end{thm}
	\end{verbatim}
\end{multicols}
\begin{multicols}{2}
%	\begin{proof}[\refthm{thm:gauss}]
	\begin{proof}
		aaa
	\end{proof}
\columnbreak
	\begin{verbatim}
	\begin{proof}
	 aaa
	\end{proof}
	\end{verbatim}
\end{multicols}

% ----------------------------
\begin{multicols}{2}
	\begin{lem}[Fermat]
		aa.
		\label{lem:fermat}
		\begin{proof}
			aaa.
		\end{proof}
	\end{lem}
\columnbreak
	\begin{verbatim}
		\begin{lem}[Fermat]
			aa.
			\label{lem:fermat}
			\begin{proof}
				aaa.
			\end{proof}
		\end{lem}
	\end{verbatim}
\end{multicols}
% ----------------------------
\begin{multicols}{2}
	\begin{cor}
		aa
	\end{cor}
\columnbreak
	\begin{verbatim}
	\begin{cor}
	 aa
	\end{cor}
	\end{verbatim}
\end{multicols}
% ----------------------------
\begin{multicols}{2}
	\begin{prop}
		aa
	\end{prop}
\columnbreak
	\begin{verbatim}
	\begin{prop}
	 aa
	\end{prop}
	\end{verbatim}
\end{multicols}
%% ----------------------------
%\begin{multicols}{2}
%	\begin{lem}
%		aa
%	\end{lem}
%\columnbreak
%	\begin{verbatim}
%	\begin{lem}
%	 aa
%	\end{lem}
%	\end{verbatim}
%\end{multicols}
% ----------------------------
\begin{multicols}{2}
	\begin{rem}
		aa
	\end{rem}
\columnbreak
	\begin{verbatim}
	\begin{rem}
	 aa
	\end{rem}
	\end{verbatim}
\end{multicols}
% ----------------------------
\begin{multicols}{2}
	\begin{notation}
		aa
	\end{notation}
\columnbreak
	\begin{verbatim}
	\begin{notation}
	 aa
	\end{notation}
	\end{verbatim}
\end{multicols}
% ----------------------------
\begin{multicols}{2}
	\begin{exmp}
		aa
	\end{exmp}
\columnbreak
	\begin{verbatim}
		\begin{exmp}
		 aa
		\end{exmp}
	\end{verbatim}
\end{multicols}



% =================
\section{Labels References}
% ----------------------------
\begin{multicols}{2}
	aaaa
	This is how to reference to them.
	\refdef{def:aa}, 
	\refthm{thm:gauss}	,
	\reflem{lem:fermat}, and
	\refsec{sec:templates}.
\columnbreak
	\begin{verbatim}
		This is how reference to them.
		\refdef{def:aa}, 
		\refthm{thm:gauss},
		\reflem{lem:aa}, and
		\refsec{sec:templates}.
	\end{verbatim}
\end{multicols}
% =================
\section{Environments}
\begin{tabular}{@{}ll@{}}
	Axiom	&\verb!ax!\\
	Corollary	&\verb!cor!\\
	Definition	&\verb!defn!\\
	Example	&\verb!exmp!\\
	Exercise	&\verb!exercise!\\
	Equation	&\verb!align*!\\
	Lemma	&\verb!lem!\\
	Notation	&\verb!notation!\\
	Proof	&\verb!proof!\\
	Proposition	&\verb!prop!\\
	Remark	&\verb!rem!\\
	Theorem	&\verb!thm!\\
%	Question	&\verb!hQuestion!\\
%	Reminder	&\verb!hReminder!\\
%	Property	&\verb!hProperty!\\
%	Notation	&\verb!hNotation!\\
%	Application	&\verb!hApplication!\\
%	Summary	&\verb!hSummary!\\
\end{tabular}



% =================
\section{Lists}
\verb!\usepackage{enumerate}!
% ----------------------------
\begin{multicols}{2}
	\begin{itemize}
		\item aa
		\item bb
	\end{itemize}
\columnbreak
	\begin{verbatim}
	\begin{itemize}
		\item aa
		\item bb
	\end{itemize}
	\end{verbatim}
\end{multicols}
% ----------------------------
\begin{multicols}{2}
	\begin{enumerate}
		\item aa
		\item bb
	\end{enumerate}
\columnbreak
	\begin{verbatim}
	\begin{enumerate}
		\item aa
		\item bb
	\end{enumerate}
	\end{verbatim}
\end{multicols}
% ----------------------------
\begin{multicols}{2}
	\begin{enumerate}[i)]
		\item aa
		\item bb
	\end{enumerate}
\columnbreak
	\begin{verbatim}
	\begin{enumerate}[i)]
		\item aa
		\item bb
	\end{enumerate}
	\end{verbatim}
\end{multicols}
% ----------------------------
\begin{multicols}{2}
	\begin{enumerate}[a)]
		\item aa
		\item bb
	\end{enumerate}
\columnbreak
	\begin{verbatim}
	\begin{enumerate}[a)]
		\item aa
		\item bb
	\end{enumerate}
	\end{verbatim}
\end{multicols}




% ======================================================================
\section{hTags}
\label{sec:hTags}

\begin{tabular}{@{}ll@{}}
%	$ new line$	&\verb!\hNewLine!\\
	$\hDefined{x}$	&\verb!\hDefined{x}!\\
%	$\hDefinedC{x}$	&\verb!\hDefinedC{x}!\\
%	$\hDefinedN{x}$	&\verb!\hDefinedN{x}!\\
	$x \hDefinitionIFF  y$	&\verb!\hDefinitionIFF!\\
	$x \hDefinitionEqual y$	&\verb!\hDefinitionEqual!\\
	$x \hDefinitionEquiv y$	&\verb!\hDefinitionEquiv!\\
%	$aaa$	&\verb!aaa!\\
\end{tabular}




% =======================================
\section{hPairs}
\label{sec:pairs}

\begin{tabular}{@{}ll@{}}
%	$\hAbs{x}$	&\verb!\hAbs{x}!\\
	$\hAbs{x}$	&\verb!\hAbs{x}!\\
%	$\hPairingBraket{x}$	&\verb!\hPairingBraket{x}!\\
%	$\hPairingParan{x}$	&\verb!\hPairingParan{x}!\\
%	$\hPairingCurly{x}$	&\verb!\hPairingCurly{x}!\\
	$\hCeil{x}$	&\verb!\hCeil{x}!\\
	$\hFloor{x}$	&\verb!\hFloor{x}!\\
	$\| x \|$	&\verb!\| x \|!\\
%	$aaa$	&\verb!aaa!\\
\end{tabular}




% =======================================
\section{hSets}
\begin{tabular}{@{}ll@{}}
	\hSoN = The natural numbers	&\verb!\hSoN!\\
	\hSoNp = Counting numbers	&\verb!\hSoNp!\\
	\hSoZ = The integers	&\verb!\hSoZ!\\
	\hSoZp = positive integers	&\verb!\hSoZp!\\
	\hSoZn = negative integers	&\verb!\hSoZn! \\
	\hSoZnn = nonnegative integers	&\verb!\hSoZnn!\\
	\hSoZnz = nonzero integers	&\verb!\hSoZnz!\\
	\hSoZnp = nonpositive integers	&\verb!\hSoZnp!\\
	\hSoQ = The rational numbers	&\verb!\hSoQ!\\
	\hSoQp = positive rational numbers	&\verb!\hSoQp!\\
	\hSoQn = negative rational numbers	&\verb!\hSoQn!\\
	\hSoQnn = nonnegative rationals	&\verb!\hSoQnn!\\
	\hSoQnz = nonzero rationals	&\verb!\hSoQnz!\\
	\hSoQnp = nonpositive rationals	&\verb!\hSoQnp!\\
	\hSoR = The real numbers	&\verb!\hSoR!\\
	\hSoRp = positive real numbers	&\verb!\hSoRp!\\
	\hSoRn = negative real numbers	&\verb!\hSoRn!\\
	\hSoRnn = nonnegative reals	&\verb!\hSoRnn!\\
	\hSoRnz = nonzero reals	&\verb!\hSoRnz!\\
	\hSoRnp = nonpositive reals	&\verb!\hSoRnp!\\
	\hSoC = The complex numbers	&\verb!\hSoC!\\
	\hSoCnz = nonzero complex numbers	&\verb!\hSoCnz!\\
	\hSoPrimes = The set of Prime Numbers	&\verb!\hSoPrimes!\\
	\hSoTruth	&\verb!\hSoTruth! \\
	$\hSoBits = \{ 0, 1 \}$	&\verb!\hSoBits!\\
	$\hSoSubsets{A}$	&\verb!\hSoSubsets{A}!\\
	$\hSoPowerSet{A}$ = $\hSoSubsets{A}$= The set of all subsets	&\verb!\hSoPowerSet{A}! \\
	$\hSoFunctions{A}{B}$=The set of all functions	&\verb!\hSoFunctions{A}{B}!\\
%	$aaa$	&\verb!aaa!\\
%	$aaa$	&\verb!aaa!\\
%	$aaa$	&\verb!aaa!\\
%	$aaa$	&\verb!aaa!\\
\end{tabular}
\end{multicols}

\end{document}
